\section{Introducción}

En esta segunda entrega de la situación problema para la unidad de formación de Analisis Metodos Matemáticos Para La Física,
se presentan los resultados de la investigación realizada en la primera etapa y los resultados de experimentación de compresión de imagenes.

Dando un poco de contexto adicional, hoy en día, enormes volúmnes de imágenes son generados y compartidos a diario, desde fotografías personales hasta imágenes atelitales o médicas.
La compresión de imágenes es un proceso util para poder disminuir el espacio de almacentamiento necesario para manejar este tipo de datos no estructurados. 
Un archivo de image sin comprimir (con algún formato de tipo RAW) almacena la información píxel por píxel y suele tener un tamaño considerable. 

Utilizando métodos de compresión es posible reducir de forma impactante el tamaño de un archivo de imagen, logrando mantener una calidad visual aceptable. Un formato popular es el formato JPEG, el cual logra tasas de compresión cercanas al 90\% (Es decir, comprime la imagen a un 10\% de su tamaño original), manteniendo una calidad de imagen considerable.
Esto se logra gracias a la redundancia de información visual y las limitaciones de percepción que tiene el ojo humano.

Existen dos tipos principales de compresión: sin pérdida (lossless), donde la imagen original se puede reconstruir exactamente igual, y con pérdida, donde se permite una leve degradación a cambio de mayores tasas de compresión.
Para las experimentaciones realizadas para esta situación problema, se probara crear un script que cree una compresión con pérdida.